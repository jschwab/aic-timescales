\documentclass[modern]{aastex62}

\usepackage{units}

\newcommand{\MESA}{{\tt MESA}}

% define some useful commands
\newcommand{\Msun}{\ensuremath{\mathrm{M}_\odot}}
\newcommand{\gcc}{\ensuremath{\mathrm{g\,cm^{-3}}}} % density units

\newcommand{\Mch}{\ensuremath{\mathrm{M}_{\rm Ch}}}

\newcommand{\Ye}{\ensuremath{Y_{\rm e}}}
\newcommand{\EF}{\ensuremath{E_{\rm F}}}

% central quantities
\newcommand{\Tc}{\ensuremath{T_{\rm c}}}
\newcommand{\Rhoc}{\ensuremath{\rho_{\rm c}}}

\newcommand{\epsnu}{\ensuremath{\epsilon_{\nu}}} % Neutrino loss rate
\newcommand{\epsnuc}{\ensuremath{\epsilon_{\mathrm{nuc}}}} % Neutrino loss rate


\input{nuclides.tex}

\usepackage{amsmath}

\begin{document}

% \author[0000-0002-4870-8855]{Josiah Schwab}
% \altaffiliation{Hubble Fellow}
% \affiliation{Department of Astronomy and Astrophysics, University of California, Santa Cruz, CA 95064, USA}
% \correspondingauthor{Josiah Schwab}
% \email{jwschwab@ucsc.edu}

% \title{}

\section{Relevant Timescales}

We define the accretion timescale as
\begin{equation}
  t_{\mathrm{accrete}} = \frac{M}{\dot{M}}
\end{equation}
and the compression timescale as
\begin{equation}
  \label{eq:tcompress}
  t_{\mathrm{compress}} = \left(\frac{d \ln \rho}{dt} \right)^{-1}~.
\end{equation}
The central density rises rapidly as one approaches the Chandrasekhar
mass and therefore, the compression timescale is significantly shorter
than the accretion timescale (by a factor $\sim 100$).


We define the electron-capture timescale as
\begin{equation}
  \label{eq:tcapture}
  t_{\mathrm{capture}} = \left(\frac{d \ln \Ye}{dt} \right)^{-1}
\end{equation}
and the heating timescale as
\begin{equation}
  \label{eq:cooling}
  t_{\mathrm{heat}} = \frac{c_P T}{\epsnuc} ~.
\end{equation}
We will primarily be interested in the exothermic electron captures,
but we will plot the absolute value of this quantity so that it
represents the cooling timescale at times when the Urca-process is
operating.


For the plots of these timescales in previous work on ECSN
progenitors, see Figure 7 in \citet{Miyaji1980}, Figure 2 in
\citet{Miyaji1987}, and Figure 9 in \citet{Takahashi2013}.


\bibliography{timescales.bib}

\end{document}

%%% Local Variables:
%%% mode: latex
%%% TeX-master: t
%%% End:
